\documentclass[letterpaper,12pt]{letter}
\usepackage{fullpage}
\usepackage{hyperref}

\setlength{\textheight}{10in} \topmargin 1in

\address{Delft University of Technology  \\  
Faculty of Electrical Engineering, \\
Mathematics, and Computer Science \\
Mekelweg 4, 2628 CD, \\ Delft, The Netherlands}
\signature{Prof.dr.ir. C. Vuik}

\begin{document}

\begin{letter}{\textbf{Letter of Intent}}

\opening{Dear SIAM Board of Trustees,}

I am writing you to inform you the interest of several graduate students and faculty members
in establishing a SIAM student chapter at Delft University of Technology in the Netherlands. 
The main purposes of the chapter are going to be the promotion of computational mathematics and scientific computing, 
as well as, the incentive of interdisiplinary activities among our PhD students and master students. 
%The aim of establishing a student chapter will therefore be to improve the scientific and personal interaction between the students. 
%This is in particular desirable in Delft because of a huge variety of nationalities among the students. 

%The chapter agenda will be in parallel with our already established series of talks. So far, we have two lecture series of a slightly different purpose running:
%\begin{itemize}
% \item \textit{tea talks:} A bi-weekly meeting where PhD students and staff members present their current research,
% \item \textit{project baNaNa:} A list of lectures for and by PhD students about scientific software and computer tools, cf. \href{http://projectbanana.github.io/}{http://projectbanana.github.io/}.
%\end{itemize}
%However, both projects are at the moment limited to the Numerical Analysis group and independent of each other. 
%The framework of a SIAM student chapter would help us to connect both lecture series and include other faculty members. 
%%Moreover, the financial support of SIAM would allow us to invite guest speakers once or twice a year.

We plan to have monthly chapter meetings or events. Typical chapter activities would include:
\begin{enumerate}
\item Launch a TU Delft SIAM student chapter Web site.
\item Organize Scientific talks about Applied Mathematics and Computer Science given by students and faculty members.
\item Organize lectures about scientific software and computer tools like: Matlab, Python, \LaTeX, among others. 
\item Promote the SIAM events and activities in internet using social networks.
%\item Coding in \textsc{Python} sessions.
%\item Version control with \textsc{git} and \textsc{github}.
%\item \LaTeX \ tips and tricks sessions - an introduction to technical writing with \LaTeX.
\item Organize social activities for members of the Chapter.
\item Participate in student day and poster session at SIAM Annual Meeting.
\item Sponsor lectures by guest speakers.
%\item Establish SIAM lounge to facilitate student interaction and collaboration.
%\item Organize talks by students or faculty.
\end{enumerate}
\newpage
Sponsor and faculty advisors are listed as the following.

\begin{tabular}{ll}
\textbf{Sponsor} &: Technical University of Delft \\
\textbf{Department} &: Faculty of Electrical Engineering, \\
&\ \ Mathematics, and Computer Science \\
\textbf{Faculty Advisor 1} &: Prof.dr.ir. C. Vuik, Professor of Numerical Analysis \\
&\ \ \href{c.vuik@tudelft.nl}{c.vuik@tudelft.nl}, +31 (0)15 27 85530 (office) \\
\textbf{Faculty Advisor 2} &: Fons Daalderop, Lecturer \\
&\ \ \href{A.G.M.Daalderop@tudelft.nl}{A.G.M.Daalderop@tudelft.nl}, +31 (0)15 27 84401 (office) \\
\end{tabular}

\bigskip

We hope that you look favorably on this application and we look forward to starting a student chapter here at Delft. Thank you for your attention.

\closing{Sincerely,}

\encl{Signed petition}

\end{letter}

\end{document}
