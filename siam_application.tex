\documentclass{article}
\usepackage{hyperref}
\usepackage{geometry}

\title{\bf SIAM Student Chapter application }
\author{Manuel Baumann}
\begin{document}
%\maketitle
\hfill \textsc{Delft Institute of Applied Mathematics}

\hfill Technical University of Delft

\hfill Mekelweg 4, 2628 CD Delft

\hfill The Netherlands

\hfill

\hfill \href{mailto:m.m.baumann@tudelft.nl}{m.m.baumann@tudelft.nl}


\vspace{2cm}
\textbf{Letter of Intent}
\\ \\
Dear SIAM Board of Trustees, \hfill \today
\\ \\
My experiences with the COSSE program are based on the academic years from September 2011 until July 2013. During these two years, I studied for the first two semesters at KTH Stockholm and later moved to TU Delft, where I completed my master thesis.
\section*{Study Background}
I did my Bachelor studies at TU Berlin in Scientific Engineering (2007-2011) as well as Mathematics (2008-2011). During the last two years of my studies, I was working as a student research assistant at the working group 'Numerics, Modeling and Differential Equations' of TU Berlin. By that time, I got in touch with the COSSE co-ordinators of TU Berlin who informed me about the program and encouraged me to apply for it. This seemed to be perfect timing because the application process for COSSE is quite time-consuming and I heard of many students that found out about COSSE towards the end of their bachelor phase and, therefore, either had time-pressure with their application for COSSE or had to wait with the application until next year. Concerning my own application procedure, I felt extremely encouraged by the professors of TU Berlin. Especially, obtaining a letter of recommendation was therefore fairly easy.

From my undergraduate education at TU Berlin, I had at every point sufficient scientific background for the master courses at both KTH and TU Delft.
\section*{First year at KTH Stockholm}
Moving to Sweden felt for me like moving to an English-speaking country. This included lectures in English at university, an English-speaking environment at the department of KTH as well as communication in English among my fellow students and at the student dorm. I was most impressed by the excellent administration at KTH: my room at the student dorm was in a very good condition and the introduction program by KTH was extremely helpful in order to get in touch with new people. Concerning my courses I got all necessary information about their content and the enrollment. At every point, I had a very clear view about my study track. Also the relation to the academic staff of KTH was very close which was especially important for project works. In my case, I formed a project group with one fellow student from Serbia and one from India in my first semester. The fact that our project was supervised by a Swedish professor completed the perfect inter-cultural experience for me.

A problem in Sweden are the relatively high living costs, especially for students with a category B scholarship. Therefore, I applied for an additional scholarship in Germany\footnote{My scholarship of 150 EUR/month was financed by the Friedrich-Ebert foundation, \url{http://www.fes.de/studienfoerderung/information-in-english}}. 
\section*{Second year at TU Delft}
In Delft, my first-year courses needed to be fit into the framework of the master program offered by TU Delft. This was done during the first study week by the master co-ordinator of the applied mathematics master at TU Delft. I experienced the COSSE community much more active in Delft which is due to the fact that TU Delft provides a joint office for all COSSE students. Moreover, Delft is a typical student city with a lot of possibilities for young people. For example, I appreciated the sport center of TU Delft a lot where most COSSE students participate.
\section*{Master thesis project}
Before I started my master project, I was studying at three different European universities. It is a huge advantage of COSSE that a student has the possibility to experience the specific scientific specialization of each university. This way, I was sure that TU Delft best corresponds to my scientific interests. The supervision of my master thesis was on a weekly basis. I had two supervisors at TU Delft and both were very flexible when deciding about the directions of my research. I finished after 8 months of project work and the results are scheduled to be published in a scientific journal.
\section*{Career after COSSE}
After completing COSSE, I had excellent job opportunities. Before finishing my thesis, I got two offers for a PhD project, one at TU Berlin and one at TU Delft. Thanks to my experiences of the last two years, I decided to continue my research at TU Delft where I am now working as a PhD student since August 2013.
 
\vspace{0.7cm} With  best regards,\\

\vspace{0.9cm}
Manuel Baumann
\end{document}
